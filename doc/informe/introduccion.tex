\section{Introducción}

Este trabajo se enmarca dentro del contexto del curso EL5004 Taller de Diseño, en el que se lleva a cabo un proyecto multidisciplinario entre estudiantes de ingeniería mecánica e ingeniería eléctrica, con el fin de lograr el acercamiento entre ambas disciplinas y el aprendizaje que esto conlleva, no solo de trabajo en equipo sino de acercar a los estudiantes a lo que corresponde a un proyecto real junto a todas sus partes: concebir, diseñar, implementar y operar.
En este caso, se propone desarrollar una teleoperación de un vehículo tipo goKart construido por alumnos del departamento de ingeniería mecánica. Esto es, comandar a distancia las acciones de giro, aceleración y freno de este vehículo. Para ello se debe realizar un diseño mecánico para el montaje de los nuevos mecanismos, como así un diseño eléctrico de la conexión y comunicación entre los actuadores. 
